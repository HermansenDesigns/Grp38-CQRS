\chapter{CQRS Implementering}
I implementationen af projekt eksemplet er der lagt særligt vægt på opdelingen af commands og query’s. Det vil sige at der har været fokus på at få adskilt read modellen fra write modellen.
Der er både lavet en implementation med ren cqrs, og en implmentation med cqrs og event sourcing. 
I command delen finde de forskellige commands for de ting man ønsker at udfører når man skal skrive til databasen. Dette er commands som at tilføje en gruppe, tilføje en user, tilmelde sig en gruppe og at omdøbe en bruger. Command’sne har sin egen context i det at dette giver mulighed for at have en database kun til write og en databse kun til read. Der er oprettet et ginerisk repository til at samling af de forskellige commands. 

I Query delen findes de forskellige querys til at read fra read databasen. Her ligger desuden de forskellige Displays af entiterne user og group. Der findes her også query context’en som udgør contexten for read databasen.\newline

Write modellen er implementeret således at den er uafhængig af entiteter. Det eneste write modellen er interesseret i er de forskellige commands. Dette giver en frihed i det at man ikke er bundet til at skulle følge en bestemt entitet, som skal stemme overens med read modellen. Dermed kan read modellens ”displays” være uafhængig af det der bliver skrevet. Det vil også sige at man ikke skal bekymre sig om read modellen når man skriver til databasen.\newline 

Ved implementeringen med ren CQRS uden event sourcing mister man en række fordele, men det er ikke kun negativt at undvære event sourcing delen. Event sourcing delen er med til at gøre implementeringen mere kompleks. En ulemperne ved implementering af CQRS uden event sourcing er at der ikke er konsistens. Dette kan være en betydelig ulempe i sistuationer hvor konsistens er nødvendigt. Dette gør det også vanskeligt at synkronisere read modellen med write modellen. Derudover ved brug uden event sourcing mister man loggen over hændelser og dermed også over eksempelvis items der er blevet added og senere removed. Dermed vil man i CQRS kun opbevare den endelige tilstand af en kollektion af items eksempelvis. 

