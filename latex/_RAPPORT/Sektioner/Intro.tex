\chapter{Introduktion}
I denne opgave skal vi undersøge et Design Pattern - som ikke har været gennemgået på kurset. 
På baggrund af interesse har vi valgt \textbf{CQRS}. Ud har opgaven, har vi undersøgt hvordan \textbf{CQRS} fungerer, og hvad der kræves for at for at implementere det for at opfylde kravene. Dette er også dokumenteret i en præsentationsvideo.
Desuden er opgaven implementeret i et Visual Studio projekt, hvilket også er demonstreret i en video.

\section{Arkitektonisk vs Design}
Generelt er ideen med at bruge et pattern, en idé til at løse et problem bedst muligt. Det kan være alt fra store komplekse koncern systemer, ned til små løsninger, som skal effektiviseres. Ideen med dem alle er at få bedre overblik over løsningen.
\newline
Software design patterns, bruges til at løse et gentagendeproblem, som kan opstå. De er derfor generelt problemløsende for udviklere, da de afhjælper med at løse komplekse problemer.
Der er mange fordele ved at bruge et sådant pattern, typisk til at lave en bedre løsning for at problem, i praktisk vil det typisk ende med en renere kodestil, som er nemmere at sætte sig ind i og forstå. Derfor er mange af disse design patterns blevet generelle at bruge, de de har vist deres fordele, når man begynder at bruge dem.
\newline
Generelt vil både Design og Arkitektoniske patterns give en ide til hvordan løsningen er, men hvor et design pattern har mere fokus på implementeringen, vil et arkitektonisk pattern have et mere abstrakt oveblik over implementeringen.
Et arkitektonisk pattern giver et overblik over system strukturen, typisk med software komponenter, og hvordan de snakker sammen. Et arkitektonisk pattern har derfor primært fokus på aspekter, som ydeenve, skalerbarhed, pålidelighed, testbarhed, vedligeholdese muligheder og mange flere faktorer. Derfor bliver disse designs typisk brugt i større komplekse situationer, hvor f.esk arbejdsbyrden skal deles op eller lignende.

\section{Termliste}
\textbf{CQRS} - Command Query Responsibility Segregation
\newline
\textbf{CRUD} - Create Read Update Delete

