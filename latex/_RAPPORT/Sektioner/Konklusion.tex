\chapter{Konklusion}
Gennem arbejdet med \textbf{CQRS} er der opnået forståelse for, hvornår et pattern bliver til et Architectual Pattern og hvornår det forbliver et Design Pattern.

\textbf{CQRS} er i sin forholdsvise enkelthed et ret simpel Architectual Pattern, hvor man i nogle situationer bliver nødt til, at løfte dette op til et mere komplekst pattern ved at udvide implementeringen med event sourcing.\newline

Eftersom \textbf{CQRS} ligger i spændingsfeltet mellen førnævte kompleksitetsgrader har det i særdeleshed været spændende at følge med i, hvornår denne linje krydses. Det har været meget indbringende at skulle definere, hvornår det forholdsvist simple \textbf{CQRS}-pattern har skulle udbygges med Event Sourcing som dels øger brugbarheden, ved nogle scenarier, men også hvordan denne udbygning kan have betydning for kompleksiteten.\newline

Selve ideen om \textbf{CQRS} er ikke omfattende at forstå og implementere, men denne har også sine begrænsninger, som kommer i spil ved de uvisse stadier på dataen. Disse uvisheder kan afhjælpes ved, at implementere \textbf{CRUD} istedet for \textbf{CQRS}, hvor Event Sourcing tilføjes. Gennem udarbejdelsen af denne opgave er det blevet defineret hvor man bør gøre hvad og hvornår. \newline

Der er ingen tvivl om at \textbf{CQRS} ligger inde med nogle fordele, som ikke kan benægtes. Blandt andet det faktum, at udviklingen af læse- og skriveoperationer kan foregå særskildt har stor betydning, da der er uligevægt i mængden af læse- og skriveoperationer. Denne særskilte implementering hjælper også på eventuelle skalerings problemer, der kan forekomme ved brug af \textbf{CRUD}.\newline 

Det har i særdeleshed været indbringende at prøve kræfter med, at implementere \textbf{CQRS}, især ved tilføjelsen af Event Sourcing. Implementeringen af \textbf{CQRS} som design pattern har medvirket til en forståelse af, hvor relativt simpel pattern'et kan være, men også hvor dette har sine udfordringer. Nogle af disse udfordringer har kunne afhjælpes ved, at lave implementeringen, hvori der også indgår Event Sourcing, men dette har givet anledning til, at skulle tænke på en andeledes måde, end man har været vant til. Eftersom denne Event drevne tankegang afviger fra den domænedrevene tankegang, som man ellers har brugt. 
Den eventdrevne tankegang har dog en kort tilvænningsperiode, da den ligger sig meget tæt op af virkelighedens events. På den måde kan der drages paralleller fra implementeringen, til den virkelige verden, hvilket har været en stor hjælp. 
