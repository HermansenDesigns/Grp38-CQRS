\chapter{Konklusion}
Gennem arbejdet med \textbf{CQRS} er der opnået forståelse for, hvornår et pattern bliver til et architectual pattern og hvornår det forbliver et design pettern.

\textbf{CQRS} er i sin forholdsvise enkelthed et ret simpel design pattern, hvor men ved i nogle situationer bliver nød til, at løfte dette op det et architectual pattern ved at udvidde implementeringen med event sourcing.

Eftersom \textbf{CQRS} ligger i spændingsfeltet mellen førnævte patterns har det i særdeleshed været spændende at følge med i, hvornår denne linje krydses. Det har været meget indbringende at skulle definere, hvornår det forholdsvist simple \textbf{CQRS}-pattern har skulle udbygges med Event sourcing som dels øger brugbarheden, ved nogle scenarier, men også hvordan denne udbygning kan have betydning for kompleksiteten.

Selve ideen om \textbf{CQRS} er ikke omfattende at forstå og implementere, men denne har også sine begrænsninger, som kommer sig i udspil ved de uvisse stadier på daten. Disse uvisheder kan afhjælped ved, enten at implementere \textbf{CRUD} istedet for \textbf{CQRS} eller at tilføje Event sourcing. Gennem udarbejdelsen af denne opgave er det blevet defineret hvor man bør gøre hvad og hvornår. 

Der er ingen tvivl om at \textbf{CQRS} ligger inde med nogle fordele som ikke kan benægtes. balnd andet det faktum, at udviklingen af Læse- og skriveoperationer kan foregå seskildt har stor betydning, da man populært sagt ofte implementere 4 gange så mange skrive operationer som læse. Denne særskilte implementering hjælper også på eventuelle skalerings problemer der kan forekomme ved brug af \textbf{CRUD}.\newline 

Det har i særdeleshed været indbringende at prøve kræfter med, at implementere dette \textbf{CQRS} både som pattern men også som det architectual pattern det kan blive til, ved tilføjelse af Event sourcing. Implementeringen af \textbf{CQRS} som design pattern har medvirket til en forståelse af, hvor relativt simpel pattern'et kan være, men også hvor dette har sine udfordringer. Nogle af disse udfordringer har kunne afhjælpes ved, at lave implementeringen, hvori der også indgår Event sourcing, men dette har givet anledning til at skulle tænke på en andeledes måde, end man har været vant til eftersom denne Event drevne tankegang afviger fra den domænedrevene tankegang som man ellers har brugt. Den eventdrevne tankegang har dog en kort tilvænningsperiode, da den ligger sig meget tæt op af virkeligheden events og at der på den måde kan drages paralleller far implementeringen til den virkelige verden, hvilket har været en stor hjælp. 

