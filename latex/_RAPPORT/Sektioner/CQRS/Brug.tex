\section{Brug}
Dette pattern bruges ligesom \textbf{CRUD}, til at skrive til og læse fra en Database. Forskelligt fra \textbf{CRUD} er læse og skrive operationerne separeret fra hinanden. Det betyder, at dette pattern er velegnet til at indgå i følgende scenarier.

\begin{itemize}  
	\item Applikationer, hvor flere læse- og skriveoperationer foregår parallelt.
	\item Applikationer, hvor der er behov for skalering
	\item Applikationer, hvor skrive- og læselogik skal udvikles uafhængigt
	\item Applikationer, hvor kompleksiteten overstiger en simpel CRUD implementering
	\item Applikationer, hvor domænet og forretningslogikken overstiger en vis kompleksitet
	
\end{itemize}

Grunden til, at \textbf{CQRS} er velegnet til disse anvendelser er, at kommandoerne kan udvikles med en grad af tilpassethed, hvilket betyder at problemer i forbindelse med sammenfletning af data kan formindskes betydeligt.
Ved brug af \textbf{CQRS}, og derved separeret udvikling af læse/skriveoperationer. Er det også muligt, at lade mere af forretningslogikken indgå i skrive operationerne, uden at dette påvirker læsningen.

\textbf{CQRS} er særdeles anvendeligt ved skalering af komplekse systemer, da det ofte er gældende, at der er uligevægt i antallet læse- og skriveoperationer. Det medvirker til, at den separerede udvikling af disse er eftertragtet.
 


